\documentclass[12pt,a4paper]{book}
 \usepackage[utf8]{inputenc}
\usepackage[spanish]{babel}
\usepackage{amsmath} 
\usepackage{amsfonts}
\usepackage{amssymb}
\usepackage{graphicx}
\usepackage[left=2cm,right=2.5cm,top=2.5cm,bottom=2.5cm]{geometry} \begin{document}
\thispagestyle{empty}

\newcommand{\HRule}{\rule{\linewidth}{0.5mm}}


 {\centering
  
 \begin{figure}[htb] \centering

\includegraphics[scale=1]{Imagenes/upt.jpg}
 \end{figure}


\large{\bf UNIVERSIDAD PRIVADA DE TACNA}\\ \vspace{0.5cm}
\large{\bf FACULTAD DE INGENIERIA }\\ \vspace{0.5cm}

\large{\bf Escuela Profesional de Ingeniería de Sistemas } \\ \vspace{1cm}

{\large {\bf Laboratorio 1 - Unidad I }}\\ \vspace{1cm}

{\large {\bf “QLIK Q” }}\\ \vspace{2cm} 

\large Curso: INTELIGENCIA DE NEGOCIOS\\ \vspace{1.5cm}
\large Docente: Ing. Patrick Cuadros Quiroga \\ \vspace{1.5cm}
\large Ticona Mamani, Alex Armando (2017057860) \\ \vspace{4cm}
\vspace{0.5cm} {\Large {\bf \textsc{Tacna - Perú} }}\\ {\Large {\bf \textsc{2019}}} \\}

	
\begin{center}
\section*{PRACTICA DE LABORATORIO 01:  VIZUALIZACION DE DATOS CON TABLEAU} 
\end{center}

\section*{Objetivo :}
Comprender la organización la información de nuestros datos de tal manera que todos los que los vean puedan comprender sus implicaciones y cómo actuar sobre ellos con claridad\\

\section*{1. Introduccion a Tableau}
\subsubsection{a.	Instalacion}

Dependiendo de la elección del producto, descargue el software en la computadora. Después de aceptar el acuerdo de licencia, puede verificar la instalación haciendo clic en el ícono de Tableau. Si aparece la siguiente pantalla, está listo para comenzar \\

\begin{center}
\includegraphics[width=12cm]{Imagenes/img1.png} 
\end{center}

\section*{2.	Comenzar}
\subsubsection{a.	Conexión a una fuente de datos}

Para comenzar a trabajar con Tableau, debemos conectar Tableau a la fuente de datos. Tableau es compatible con muchas fuentes de datos. Las fuentes de datos compatibles con Tableau aparecen en el lado izquierdo de la pantalla inicial. Algunas fuentes de datos de uso común son Excel, archivos de texto, bases de datos relacionales o incluso en un servidor. También se puede conectar a una fuente de base de datos en la nube como Google Analytics, Amazon Redshift, etc. \\
\begin{center}
\includegraphics[width=12cm]{Imagenes/img2.png}
\end{center}

En la pestaña Hojas, se verán tres hojas: Pedidos, Personas y Devoluciones. Sin embargo, nos centraremos solo en los datos de los pedidos. Haga doble clic en Hoja de pedidos y se abrirá como una hoja de cálculo. \\
\begin{center}
\includegraphics[width=12cm]{Imagenes/img3.png}
\end{center}
Observamos que las primeras tres filas de datos se ven un poco diferentes y no están en el formato deseado. Aquí utilizamos el intérprete de datos , también presente en la pestaña Hojas. Al hacer clic en él, obtenemos una hoja con un formato agradable 

\begin{center}
\includegraphics[width=12cm]{Imagenes/img4.png}
\end{center}

\subsubsection{b.	Crear una vista}
Vaya a la hoja de trabajo. Haga clic en la pestaña Sheet 1 en la parte inferior izquierda del espacio de trabajo del cuadro.
\begin{center}
\includegraphics[width=12cm]{Imagenes/img5.png}
\end{center}
Una vez que esté en la hoja de trabajo, desde Dimensions de bajo del panel Datos, arrastre Order Dateal estante Columna.\\
\begin{center}
\includegraphics[width=12cm]{Imagenes/img6.png}
\end{center}
Del mismo modo, desde la Measures pestaña, arrastre el Sales campo al estante Filas. \\
\begin{center}
\includegraphics[width=12cm,height=12cm]{Imagenes/img7.png}
\end{center}
\subsubsection{c.	Refinando la vista}
Category está presente en el panel Dimensiones. Arrástrelo al estante de columnas y colóquelo junto a YEAR(Order Date). El Category debe ser colocado a la derecha de Year. Al hacerlo, la vista cambia inmediatamente a un tipo de gráfico de barras desde una línea. El gráfico muestra el total Sales de cada Product año. \\
\begin{center}
\includegraphics[width=12cm,height=6cm]{Imagenes/img8.png}

\includegraphics[width=12cm,height=5cm]{Imagenes/img9.png}
\end{center}
El gráfico de barras también se puede mostrar horizontalmente en lugar de verticalmente. Haga clic Swapen la barra de herramientas para el mismo. \\
\begin{center}
\includegraphics[width=12cm,height=6cm]{Imagenes/img10.png}
\end{center}
La vista por encima de Niza los espectáculos sales de category, por ejemplo, muebles, equipos de oficina, y la tecnología. También podemos inferir que las ventas de muebles están creciendo más rápido que las ventas de suministros de oficina, excepto en 2016. Por lo tanto, sería prudente centrar los esfuerzos de ventas en muebles en lugar de suministros de oficina. Pero los muebles son una categoría amplia y se componen de muchos elementos diferentes. ¿Cómo podemos identificar qué mueble está contribuyendo a las ventas máximas?


\begin{center}
\includegraphics[width=12cm,height=5cm]{Imagenes/img11.png}
\end{center}

\section*{3.	Enfatizando los resultados}
\subsubsection{a.	Agregar filtros a la vista}

En el panel Datos, en Dimensiones, haga clic con el botón derecho en Fecha de pedido y seleccione Mostrar filtro. Repita también para el campo Sub-> categoría

\begin{center}
\includegraphics[width=12cm,height=7cm]{Imagenes/img12.png}
\end{center}

\subsubsection{b.	Agregar Colores a la vista}

En el panel Datos, en Dimensiones, haga clic con el botón derecho en Fecha de pedido y seleccione Mostrar filtro. Repita también para el campo Sub-> categoría

\begin{center}

\includegraphics[width=12cm,height=12cm]{Imagenes/img14.png}
\end{center}

\subsubsection{c.	Resultados clave}
En la vista, en la Sub-Category tarjeta de filtro, desactive todas las casillas excepto Bookcases, Tables, y Machines. Esto saca a la luz un hecho interesante. Mientras que en algunos años, las librerías y las máquinas fueron realmente rentables. Sin embargo, en 2016, Machines dejó de ser rentable.
\begin{center}

\includegraphics[width=12cm,height=10cm]{Imagenes/img15.png}
\end{center}
Seleccione Allen la Sub-Category tarjeta de filtro para mostrar todas las subcategorías nuevamente.
\begin{center}

\includegraphics[width=12cm,height=10cm]{Imagenes/img16.png}
\end{center}

Desde Dimensiones, arrastre Regional Rowsestante y colóquelo a la izquierda de la pestaña Suma (Ventas). Observamos que las máquinas en el sur están reportando un beneficio negativo mayor en general que en sus otras regiones.
\begin{center}

\includegraphics[width=12cm,height=10cm]{Imagenes/img17.png}
\end{center}


Démosle ahora un nombre a la hoja. En la parte inferior izquierda del espacio de trabajo, haga doble clic Sheet 1y escriba Sales by Product and Region \\
\begin{center}

\includegraphics[width=12cm,height=10cm]{Imagenes/img18.png}
\end{center}
Para conservar la vista, Tableau nos permite duplicar nuestra hoja de trabajo para que podamos continuar en otra hoja desde donde la dejamos.

\begin{center}

\includegraphics[width=12cm,height=10cm]{Imagenes/img20.png}
\end{center}
Por último, no olvide guardar los resultados seleccionando File > Save As. Nombremos nuestro libro de trabajo como Regional Sales and Profits

\begin{center}

\includegraphics[width=12cm,height=5cm]{Imagenes/img21.png}
\end{center}

\subsection*{4.	Vista de mapa}
\subsubsection{a.	Crear una vista de mapa}

Crea una nueva hoja de trabajo\\
Agregue State y Country en el panel Datos a Detail en la tarjeta Marcas. Obtenemos la vista del mapa.

\begin{center}

\includegraphics[width=12cm,height=5cm]{Imagenes/img22.png}
\end{center}

Arrastre Region a la Filters estantería y luego filtre hacia abajo Southsolo. La vista del mapa ahora se acerca solo a la región Sur y una marca representa cada estado

\begin{center}

\includegraphics[width=12cm,height=5cm]{Imagenes/img23.png}
\end{center}

Arrastre la Sales medida a la Colorpestaña de la tarjeta Marcas. Obtenemos un mapa relleno con los colores que muestra el rango de ventas en cada estado

\begin{center}
\includegraphics[width=12cm,height=5cm]{Imagenes/img24.png}
\end{center}

Podemos cambiar el esquema de color haciendo clic Color en la tarjeta Marcas y seleccionando Edit Colors. Podemos experimentar con las paletas disponibles.\\

Observamos que Florida se está desempeñando mejor en ventas. Si pasamos el cursor sobre Florida, muestra un total de 89,474 USD en ventas, en comparación con Carolina del Sur, por ejemplo, que tiene solo 8,482 USD en ventas. Evaluemos el rendimiento a Profit estas alturas, ya que las ganancias son un mejor indicador que las ventas por sí solas\\

Arrastre Profit hacia Color en la tarjeta Marcas. Ahora vemos que Tennessee, Carolina del Norte y Florida tienen ganancias negativas, aunque parecía que les estaba yendo bien en Ventas. Cambiar el nombre de la hoja como Profit Map
\begin{center}
\includegraphics[width=12cm,height=5cm]{Imagenes/img26.png}
\end{center}

\subsubsection{b.	Entrar en los detalles}

Duplique la hoja de trabajo Mapa de beneficios y asígnele el nombre Gráfico de barras de beneficios negativo\\

Haga clic Show Meen la hoja de trabajo Gráfico de barras de ganancias negativas . Show Mepresenta el número de formas en que se puede trazar un gráfico entre los elementos\\
\begin{center}
\includegraphics[width=5cm,height=12cm]{Imagenes/img27.png}
\end{center}
Podemos seleccionar más de una barra a la vez simplemente haciendo clic y arrastrando el cursor sobre ellas. Queremos centrarnos únicamente en los tres estados, es decir, Tennessee, Carolina del Norte y Florida. Por lo tanto, solo seleccionaremos las barras correspondientes\\

\begin{center}
\includegraphics[width=12cm,height=5cm]{Imagenes/img28.png}
\end{center}

Ahora vemos que Jacksonville y Miami, Florida; Burlington, Carolina del Norte; y Knoxville y Memphis, Tennessee, son las ciudades con peor desempeño en términos de ganancias. Hay otra marca en la vista, Jacksonville, Carolina del Norte, que no pertenece aquí ya que tiene ventas rentables. Esto significa que hay un problema en el filtro que aplicamos. Aceptaremos la ayuda de Tableau Order of Operations \\

En el estante Filtros, haga clic con el botón derecho en el conjunto Inclusiones (país, estado) y seleccione Add to Context. Encontramos que ahora Concord ( Carolina del Norte ) aparece a la vista mientras Miami ( Florida ) han desaparecido. Esto tiene sentido ahora.\\
\begin{center}
\includegraphics[width=12cm,height=8cm]{Imagenes/img29.png}
\end{center}

Pero Jacksonville ( Carolina del Norte ) todavía está presente, lo cual es incorrecto. En el estante Filas, haga clic en el icono con forma de más en la Citypestaña para profundizar en el nivel de Código postal. Haga clic con el botón derecho en el código postal de Jacksonville, NC, 28540, y luego seleccione Excludepara excluir Jacksonville manualmente .
\begin{center}
\includegraphics[width=12cm,height=8cm]{Imagenes/img30.png}
\end{center}
Arrastre Código postal del estante Filas. Esta es la vista final
\begin{center}
\includegraphics[width=12cm,height=8cm]{Imagenes/img31.png}
\end{center}

\subsubsection{c.	Resultados clave}
Arrastre Sub-Categorya las Filas para profundizar más \\

Del mismo modo, arrastre Profithacia Coloren la tarjeta Marcas. Esto nos permite detectar rápidamente productos con beneficios negativos

\begin{center}
\includegraphics[width=12cm,height=8cm]{Imagenes/img32.png}
\end{center}

Haga clic derecho en Order Datey seleccione Show Filter. Parece que las máquinas, las tablas y las carpetas funcionan mal. ¿Entonces, qué debemos hacer? ¿Una solución sería detener la venta de estos productos en Jacksonville, Concord, Burlington, Knoxville y Memphis? Verifiquemos si nuestra decisión es correcta \\

Regresemos a la Profit Mappestaña de la hoja creada anteriormente\\
Ahora, haga clic en el Sub-Categorycampo para seleccionar la Show Filteropción\\

Nuevamente, haga clic en Order Datey seleccione Show Filter. Del filtro, eliminemos los elementos que creemos que contribuyen al beneficio negativo. Por lo tanto, desmarque las casillas frente a Carpetas, Máquinas y Tablas, respectivamente. Ahora solo nos quedan las entidades lucrativas. Esto muestra que las entidades como los aglutinantes, las máquinas y las tablas en realidad estaban causando pérdidas en algunas áreas y teníamos razón en nuestros hallazgos\\

\begin{center}
\includegraphics[width=12cm,height=8cm]{Imagenes/img34.png}
\end{center}

\subsection*{5.	Tablero}
\subsubsection{a.	Crear un tablero}
Haga clic en el New dashboard botón\\
Arrastra Sales in the South al tablero vacío
\begin{center}
\includegraphics[width=12cm,height=8cm]{Imagenes/img35.png}
\end{center}

Arrastre Profit Map al tablero y suéltelo encima de Ventas en la vista Sur. Ambas vistas se pueden ver a la vez. Para poder presentar los datos de manera que otros puedan entenderlos, podemos organizar el tablero a nuestro gusto

\begin{center}
\includegraphics[width=12cm,height=8cm]{Imagenes/img36.png}
\end{center}

Explore y experimente. En la visualización a continuación, podemos filtrar el Sales Southmapa para ver los productos que se venden solo en Carolina del Norte. Luego, podemos explorar fácilmente las ganancias anuales.

\begin{center}
\includegraphics[width=12cm,height=8cm]{Imagenes/img37.png}
\end{center}
-	Cambie el nombre del panel a Regional Sales and Profit.

\subsection*{6.	Historia}
\subsubsection{a.	Construyendo Historia}

\begin{center}
\includegraphics[width=12cm,height=8cm]{Imagenes/img38.png}
\end{center}
-	Edite el texto en el cuadro gris sobre la hoja de trabajo. Este es el título. Nómbrelo como Sales and profit by year

\begin{center}
\includegraphics[width=12cm,height=8cm]{Imagenes/img42.png}
\end{center}

Las historias son bastante específicas. Aquí contaremos una historia sobre la venta de máquinas en Carolina del Norte. En el panel Historia, haga clic en Duplicate para duplicar el primer título, o incluso puede crear uno nuevo.\\

En el Sub-Category, select solo filtro Machines. Esto ayuda a medir las ventas y los beneficios de las máquinas por año\\

Cambie el nombre del título a Machine sales and profit by year

\subsubsection{b.	Hacer una Conclusion}
En el panel Historia, seleccione Blank. Arrastre el panel ya creado Regional Sales and Profit al lienzo \\

Subtitúlelo como Low performing items in the South

\begin{center}
\includegraphics[width=12cm,height=8cm]{Imagenes/img43.png}
\end{center}
Seleccione Duplicate para crear otro punto de la historia con el panel de ganancias regionales. Seleccione Carolina del Norte en el gráfico de barras, ya que estamos interesados en mostrar más al respecto


\begin{center}
\includegraphics[width=12cm,height=8cm]{Imagenes/img45.png}
\end{center}

Seleccione Todos los años\\

Agregar un título para mayor claridad, como, Profit in NC : 2013-2016\\

Seleccione cualquier año como 2014. Agregue un título, por ejemplo, Profit in NC : 2014y luego haga clic en la pestaña Duplicar. Repita el mismo paso para todos los años restantes

\begin{center}
\includegraphics[width=12cm,height=8cm]{Imagenes/img46.png}
\end{center}

Haz clic en el modo de presentación y deja que se story desarrolle

\begin{center}
\includegraphics[width=12cm,height=8cm]{Imagenes/img47.png}
\end{center}
\end{document}


